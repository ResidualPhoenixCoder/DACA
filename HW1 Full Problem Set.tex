\documentclass[fleqn]{article}
\usepackage[margin=0.24in]{geometry}

\usepackage{amsmath,amsfonts}
\usepackage{algorithm,algpseudocode}
\usepackage{enumitem}
\usepackage{tikz}
\setdescription{leftmargin=\parindent,labelindent=\parindent}
\title{CS 344: Homework 1 }
\author{Mark Labrador, Janelle Barcia,Conrado Uraga}
\date{\today}

\begin{document}
\maketitle
\section{Big Oh Comparison}         
\textbf{1. f(n) = $\sqrt{2^{7n}}$, g(n) = lg(7$^{2n}$)}\\
$\sqrt{128^{n}}$ = lg(49$^{n}$)\\
${128^{n/2}}$ = lg(49) * n \\
lg(${128^{n/2}}$) = lg(lg(49) * n)\\
$\frac{lg(128)}{2}$ * n = lg(lg(49)) + lg(n)\\
Any polynomial dominates any logarithm.\\\\
$\Omega$(g(n)) = \{f(n): there exists positive constants c and  $n_{0}$, such that 0 $\le$ c * g(n) $\le$ f(n) for all n $\ge$ $n_{0}$\\
$\frac{lg(128)}{2}$ * n $\ge$ lg(lg(49)) + lg(n) $\ge$ 0 for all n $\ge$ 0\\ 
with multiplicative terms omitted: n $\ge$ lg(n) $\ge$ 0 for all n $\ge$ 0\\ 
\textbf{ANSWER: f(n) = $\Omega$(g(n))}\\\\
\textbf{2. f(n) = $2^{n*ln(n)}$, g(n) = n!}\\
By Stirling's Approximation, n! = $\sqrt{2\pi n}$ * $(\frac n e)^{n}$\\
$2^{n*ln(n)}$  = $\sqrt{2\pi n}$ * $(\frac n e)^{n}$\\
lg($2^{n*ln(n)}$)  = lg($\sqrt{2\pi n}$ * $(\frac n e)^{n}$)\\
n*ln(n)*lg(2) = n*lg($\sqrt{2\pi n}$ * $(\frac n e)$)\\
n*ln(n) = n*lg($\sqrt{2\pi n}$ * $(\frac n e)$)\\\\
$\frac{f(n)}{g(n)}$ is bounded:\\
$\frac{n*ln(n)}{n*lg(\sqrt{2\pi n} *(\frac{n}{e}))}$ $<$ 1 as n approaches $\infty$ \\\\
$\frac{g(n)}{f(n)}$ is bounded:\\
$\frac{n*lg(\sqrt{2\pi n} * (\frac n e))}{n*ln(n)}$ $<$ 2 as n approaches $\infty$ \\\\
As both functions are bounded as n grows to $\infty$, then \textbf{f(n) = $\Theta$g(n).}\\
\textbf{ANSWER: f(n) = $\Theta$g(n).}\\\\
\textbf{3. f(n) = lg(lg * n), g(n) = lg*(lgn)}\\
We can use the identity: lg*n = 1+lglg(n)\\
lg(1 + lglg(n) = 1 + lglglg(n)\\
$2^{lg(1 + lglg(n)} = 2^{1 + lglglg(n)}$\\
1 + lglg(n) = 2lglg(n)\\
1 + lglg(n) = lglg(n) + lglg(n)\\
2lglg(n) $>$ lglg(n)\\
Therefore: O(g(n)) = \{f(n): there exists positive constants c and  $n_{0}$, such that 0 $\le$ f(n) $\le$ c * g(n) for all n $\ge$ $n_{0}$\\
0 $\le$ f(n) $\le$ g(n) for all n $>$ $4$\\
\textbf{ANSWER: f(n) = O(g(n))}\\\\
\textbf{4. f(n) = $\frac{lgn^{2}}{n}$, g(n) = lg*n}\\
As we are concerned with large n as input grows when examining run time, we can make the substitution lg*n = 1 + lg(lgn) when n $>$ 2.\\
$\frac{lgn^{2}}{n}$ = 1 + lg(lgn)\\
$\frac{2lgn}{n}$ = 1 + lg(lgn)\\
if we substitute k for lg(n), we can represent this function as:\\
$\frac{2k}{n}$, where k $<$n for all n\\
$\frac{2k}{n}$ = 1 + lg(k)\\
lg(n) $\ge$ k * $\frac{1}{n}$ for all n $\ge$ 1, as $\frac{1}{n}$ is a limit that approaches 0 when taken to infinity and is not increasing.\\
Therefore: O(g(n)) = \{f(n): there exists positive constants c and  $n_{0}$, such that 0 $\le$ f(n) $\le$ c * g(n) for all n $\ge$ $n_{0}$\\
0 $\le$ f(n) $\le$ g(n) for all n $>$ $2$\\
\textbf{ANSWER: f(n) = O(g(n))}\\\\
\textbf{5. f(n) = 2$^{n}$, g(n) = n$^{lgn}$}\\
2$^{n}$ = n$^{lgn}$\\
lg(2$^{n}$) = lg(n$^{lgn}$)\\
n * lg(2) = lgn * lgn\\
n = lg$^{2}$n\\
Any polynomial dominates any logarithm.\\
$\Omega$(g(n)) = \{f(n): there exists positive constants c and  $n_{0}$, such that 0 $\le$ c * g(n) $\le$ f(n) for all n $\ge$ $n_{0}$\\
n $\ge$ lg$^{2}$n $\ge$ 0 for all n $>$ 1\\
\textbf{ANSWER: f(n) = $\Omega$(g(n))}\\\\
\textbf{6. f(n) = 2$^{\sqrt{lgn}}$, g(n) = n(lgn)$^{3}$}\\
2$^{\sqrt{lgn}}$ = n(lgn)$^{3}$ \\
lg(2$^{\sqrt{lgn}}$) = lg(n*(lgn)$^{3}$)\\
$\sqrt{lgn}$ * lg(2) = 3 * lg(n*(lgn))\\
$\sqrt{lgn}$ = 3 * lg(n*(lgn))\\
Linearithmic functions grow faster than linear functions (n $<$ n*lgn),  because linearithmic means the log function performed n times, therefore 3 * lg(n*(lgn)) $\ge$ $\sqrt{lgn}$ $\ge$ 1.\\
\textbf{ANSWER: f(n) = O(g(n))}\\\\
\textbf{7. f(n) = e$^{cosx}$, g(n) = lgn}\\
The function e$^{cosx}$ always oscillates between the values $e^{cos(0)}$ $\approx$ 2.71 and $\frac{1}{e}$ $\approx$ 0.367, therefore it does not continuously increase. g(n) increases, which means the function will continuously grow for greater values of n.\\
O(g(n)) = \{f(n): there exists positive constants c and  $n_{0}$, such that 0 $\le$ f(n) $\le$ c * g(n) for all n $\ge$ $n_{0}$\\
lgn$\ge$ e$^{cosx}$ $\ge$ 0 for all n $\ge$ 0.\\
\textbf{ANSWER: f(n) = O(g(n))}\\\\
\textbf{8. f(n) = lgn$^{2}$, g(n) = (lgn)$^{2}$}\\
lgn$^{2}$ = (lgn)$^{2}$\\
2*lgn = lgn*lgn\\\\
$\frac{2*lgn}{lgn*lgn}$ $<$ 1 for all n $\ge$ 8 (roughly e$^{2}$)\\
$\frac{lgn*lgn}{2*lgn}$ is unbounded and approaches $\infty$\\
The e
O(g(n)) = \{f(n): there exists positive constants c and  $n_{0}$, such that 0 $\le$ f(n) $\le$ c * g(n) for all n $\ge$ $n_{0}$\\
lgn*lgn $\ge$ 2*lgn $\ge$ 0 for all n$\ge$0\\
\textbf{ANSWER: f(n) = O(g(n)).}\\\\
\textbf{9. f(n) = $\sqrt{4n^{2} - 12 n + 9}$, g(n) =  n$^{\frac{3}{2}}$}\\
$\sqrt{4n^{2} - 12 n + 9}$ = $\sqrt{n^{3}}$\\
$\sqrt{4n^{2} - 12 n + 9}^{2}$ = $\sqrt{n^{3}}^{2}$\\
$4n^{2} - 12 n + 9 = n^{3}$\\
The exponent with the greatest value always dominates.\\\\
O(g(n)) = \{f(n): there exists positive constants c and  $n_{0}$, such that 0 $\le$ f(n) $\le$ c * g(n) for all n $\ge$ $n_{0}$\\
$n^{3} \ge 4n^{2} - 12 n + 9 \ge $0 for all n$ \ge$ 0\\ 
\textbf{ANSWER: f(n) = O(g(n))}\\\\
\textbf{10. f(n) = $\sum\nolimits_{k=1}^n k$, g(n) = (n + 2)$^{2}$}\\
$\sum\nolimits_{k=1}^n k$ = $\frac{n(n+1)}{2}$ for all natural numbers n shown by mathematical induction.\\
$\frac{n(n+1)}{2}$ = (n + 2)$^{2}$\\
$\frac{1}{2} * (n^{2}$ + n) =  $n^{2}$ + 2n + 4\\
$n^{2}$ + n =  2$n^{2}$ + 4n + 8\\
$\frac{f(n)}{g(n)}$ is bounded:\\
$\frac{n^{2} + n}{2n^{2} + 4n + 8}$ $\le$ $\frac{1}{2}$ as n approaches $\infty$ \\\\
$\frac{g(n)}{f(n)}$ is bounded:\\
$\frac{2n^{2} + 4n + 8}{n^{2} + n}$ $\le$ 2 as n approaches $\infty$ \\\\
As both functions are bounded as n grows to $\infty$, then \textbf{f(n) = $\Theta$g(n).}\\
\textbf{ANSWER: f(n) = $\Theta$g(n).}\\\\
\section{Runtime of Number Theoretic Algorithm}
\textbf{Algorithm 1: Number$\_$Theoretic$\_$Algorithm(integer n)}\\
line 1: \textbf{N $\leftarrow$ Random$\_$Sample(0,2$^{n}-1$);} This runs at O(n) from bit shift 2 exponent\\
line 2: \textbf{if N is even then}  O(1)\\
line 3: \textbf{N $\leftarrow$ N + 1 ;}  O(1)\\
line 4: \textbf{m $\leftarrow$ N mod n;} O(n$^{2}$) because modular operaton\\
line 5: \textbf{for j $\leftarrow$ 0 to m do} linear loop that will execute from 0 to N-1, O(n)\\
line 6: \textbf{if Greatest$\_$Common$\_$Divisor(j,N) $\neq$ 1 then} GCD is O(n$^{3})$ according to DPV\\
line 7: \textbf{return FALSE;} O(1)\\
line 8: \textbf{Compute x,z so that N-1 = 2$^{z}$ * x and x is odd;} this takes O(log n), shift until the last binary number is 1(this shows it's odd), then the number of shifts is z. N-1 is known previously and the value of x is the new binary number value. The number of bits is N-1 so O(log N-1), which goes to O(log n)\\
line 9: \textbf{${y_0}$ $\leftarrow$ (N - 1 - j)$^{x}$ mod N;} modular exponentiation O(n$^{3}$)\\
line 10: \textbf{for i $\leftarrow$ 1 to m do} linear loop that will execute from 0 to N-1, O(n)\\
line 11: \textbf{${y_i}$ $\leftarrow$ ${y_{i-1}}^{2}$ mod N;} O(n$^2$) because modular operation costs O(n$^2$) and y*y costs O(n$^2$), O(n$^2$) + O(n$^2$)\\
line 12: \textbf{${y_i}$ $\leftarrow$ ${y_i}$ + ${y_{i - 1}}$ mod N;} O($n^{2}$) + O(n)\\
line 13: \textbf{if Low$\_$Error$\_$Test${y_m}$ == FALSE} prime test is O(n$^{3}$) according to DPV\\
line 14: \textbf{return FALSE;} O(1) \\
line 15: \textbf{return TRUE;} O(1)\\
The running time for lines 1-4 are: O(n) + O(1) + O (1) + O(n$^{2}$)\\
Lines 5-14 are contained within a loop that runs at most (n-1) times. The run time of these combined lines is: O(n) + O(n$^{3}$) + O(1) + O(log n) + O(n$^{3}$) + O(n) + O(n$^2$) + O(n$^2$) + O(n$^3$) + O(1)\\
These lines all run n times so the runtime here is n*(O(n) + O(n$^{3}$) + O(1) + O(log n) + O(n$^{3}$) + O(n) + O(n$^2$) + O(n$^2$) + O(n$^3$) + O(1))\\
giving a combination of: O(n) + O(1) + O (1) + O(n$^{2}$) + n*(O(n) + O(n$^{3}$) + O(1) + O(log n) + O(n$^{3}$) + O(n) + O(n$^2$) + O(n$^2$) + O(n$^3$) + O(1))\\
The most expensive operation is n*O(n$^3$), giving a total runtime of O(n$^4$).\\\\
 The lower bound of the height of a tree data structure $T_m^{N}$, where every node has at most m children and the tree has at most N nodes, occurs when a complete tree is formed (which is when every level has m children, and the last level has at most m children).\\
 \section{Asymptotic Tree Analysis }
 \paragraph{Problem 3a -} ~\\
The height of the lower bound can be shown by comparing it to the total number of nodes at every level: 1 + m + m$^{2}$ + m$^{3}$ + ... + m$^{h-1}$ = $\frac{m^{h} - 1}{m - 1}$ = N\\
$\frac{m^{h} - 1}{m - 1}$ = N\\
$m^{h}$ = N * (m - 1) + 1\\
$m^{h}$ = N * (m - 1 + $\frac{1}{N}$)\\
$\log_m$($m^{h}$) = $\log_m$(N * (m - 1 + $\frac{1}{N}$))\\
h  = $\log_m$(N) + $\log_m$(m - 1 + $\frac{1}{N}$))\\
Therefore the total height can be computed by $\lceil \log_m N (m-1) \rceil$.
\paragraph{Problem 3b -} ~\\
To show the asymptotic behavior of two functions, we can take a limit of both of the functions to see what they are approaching when tending to $\infty$.\\
h = $\log_m$(N) + $\log_m$(m - 1) = $\log_m$(N(m - 1))\\
f(n) =  $\log_m$(N(m - 1))\\
g(n) = $\log_{m^{'}}$(N(m$^{'}$ - 1))\\
$\lim_{n \rightarrow \infty}$ $\frac{f(n)}{g(n)}$\\
$\lim_{n \rightarrow \infty} \frac{\log_m(N(m - 1))}{\log_{m^{'}}(N(m^{'} - 1))}$\\
$\lim_{n \rightarrow \infty} \frac{\log_m(Nm - N))}{\log_{m^{'}}(Nm^{'} - N))}$\\
$\lim_{n \rightarrow \infty} \frac{\log_m(m)}{\log_{m^{'}}(m^{'})}$\\
if m $>$ m$^{'}$, then \\
if m $<$ m$^{'}$, then \\
It must be noted that this only holds true when, m $>$ 1 because (1-1) is 0 and the log(0) does not exist.
\paragraph{Problem 3c -} ~\\
\begin{align*}
\text{We are given a similar recursive algorithm for Modular exponentiation in DPV 1.2.2. We are shown}
\end{align*}
\paragraph{Problem 4a - Multiplicative Inverses} ~\\
\begin{align*}
2^{902} \equiv\ 2^{6\cdot150 + 2} \equiv\ 2^{2}2^{6\cdot150} \equiv\ 2^{2}(2^{6})^{150} \equiv\ 2^{2}(1)^{150} \equiv\ 2^2 \equiv\ 4\ \textrm{mod}\ 7\text{, by Fermat's Little Theorem.}
\end{align*}

\paragraph{Problem 4b - Multiplicative Inverses} ~\\
\begin{itemize}
	\item $11y \equiv\ 1\ \textrm{mod}\ 120$, $y=11$
	\item $13y \equiv\ 1\ \textrm{mod}\ 45$, $y=7$
	\item $35y \equiv\ 1\ \textrm{mod}\ 77$, $y$, does not exist because 35 and 77 are not relatively prime.
	\item $9y \equiv\ 1\ \textrm{mod}\ 11$, $y=5$
	\item $11 \equiv\ 1\ \textrm{mod}\ 1111$, $y$ does not exist because 11 and 1111 are not relatively prime.
\end{itemize}

\paragraph{Problem 4c - NO ANSWER}

\paragraph{Problem 5a - Greatest Common Divisor}
True.
\begin{align*}
\gcd{(x,y)} &= \gcd{(x, x+y)}\\
&= \gcd{(x+y, x+x+y)}\\
&= \gcd{(2x+y+x+y, 2x+y)}\\
&= \gcd{(3x+2y,2x+y+3x+2y)}\\
&= \gcd{(3x+2y,5x+3y)}
\end{align*}

\paragraph{Problem 5b - Greatest Common Divisor}
This will be proved using the property that $\gcd(a,b) = \gcd(b, a mod b)$.\\

Assume that $1 \le i,j \le n$, $i \neq j$, and $i < j$.  Observe the following: \\
\begin{align*}
s_j\ \equiv\ 1 \textrm{mod}\ s_i
\end{align*}

This is because $s_j= 1 + \prod\limits_{l=0}^{j-1} s_l$, which means $s_i$ is contained in the term $\prod\limits_{l=0}^{j-1} s_l$.  So applying the mod operator with $s_i$ will cause this term to disappear, and leave $1$ as the remaining term.  This implies, $\gcd(s_j,s_i)=\gcd(s_i, s_j\ \text{mod}\ s_i)=\gcd(s_i, 1)=1$.  Therefore, all $s_k$ are relatively prime.

\pagebreak
\paragraph{Problem 6a - Universal Hashing}
Suppose $h \in H$, where $H$ is the family of hashing functions, and $m \in M$, where $M$ is the set of all 8 x 32 binary matrices.  If a 32-bit integer is selected and converted to a 32 x 1 matrix called, $y$, then the following operation is performed,

\begin{align*}
h(y) &= m \cdot y\ \textrm{mod}\ 2
\end{align*}

\noindent
Let $s_i=\sum\limits_{j=0}^{31} m_{i,j}y_{j}\ \textrm{mod}\ 2$, where $m_{i,j}$ is the entry of the i\textsuperscript{th} row and j\textsuperscript{th} column of the matrix $M$ and $y_{j}$ is the j\textsuperscript{th} row of the $y$ matrix.\\

\noindent
After $h(m, y)$ is performed, the resulting 8-bit vector call, $H$ has the entries $s_i$ for $i = 0, \ldots, 7$.  To determine the probability of hashing to any one slot of the 256 possible slots, the probability of hashing to any one 8-bit number is what needs to be determined, bit-by-bit.\\

\noindent
Suppose two distinct integers are chosen, $y_1$ and $y_2$ such that their last bit differs.  So to compute the probability of picking a row like this, the following relationship is established.\\

\noindent
Let E be the event where the last bit of each of column of $m$ is chosen such that the relationship below holds.
\begin{align*}
\sum\limits_{j=0}^{30} m_{i,j}\left(y_{2j} - y_{1j}\right)\ &\equiv\ m_{i,31}(y_{2(31)} - y_{1(31)})\ \textrm{mod}\ 2\\
Pr\{h(m, y_1) = h(m, y_2)\} &= Pr\{\textrm{E\}}
\end{align*}

\noindent
Since 2 is prime and $y_{2j} \neq y_{1j}$, there is an unique inverse for $y_{2(31)} - y_{1(31)}$ that is either 0 or 1.  So $Pr\{E\}= \frac{1}{2}$.\\

\noindent
This occurs for every row of the matrix H.  So the probability of getting an 8-bit matrix H is the product of its parts.  This means, $Pr\{$Hashing to 1 out of 256 slots$\}=Pr\{E\}=(\frac{1}{2})^{8}=\frac{1}{256}$.  Therefore, the family of functions, $H$ is universal.

\paragraph{Problem 6b - Random Bits}
This family required 256 random bits.

\pagebreak
\paragraph{Problem 7a} ~\\
Finding the integers that are their own inverses is the same as asking, $x^2 \equiv\ 1\ \textrm{mod}\ n$.  This gives the following,

\begin{align*}
x^2 &\equiv\ 1\ \textrm{mod}\ n\\
x^2 - 1 &\equiv\ 0\ \textrm{mod}\ n\\
(x+1)(x-1) &\equiv\ 0\ \textrm{mod}\ n\\
x+1 &\equiv\ 0\ \textrm{mod}\ n \rightarrow x \equiv\ -1\ \equiv\ n-1\ \textrm{mod}\ n\\
x-1 &\equiv\ 0\ \textrm{mod}\ n \rightarrow x \equiv\ 1\ \textrm{mod}\ n
\end{align*}

So the integers that are their own inverses are $n-1$ and $1$ modulo $n$ for $x$ in the range of $0$ to $n-1$.
\paragraph{Problem 7b} ~\\
For $p=2$, $(p-1)! \equiv\ (2-1)\ \equiv\ 1\ \equiv\ -1 \textrm{mod}\ 2$.\\\\
Suppose $p > 2$ and $p$ is prime.  Then $b \in B = \{0, 1, 2, \ldots, p-1\}$ has a multiplicative inverse modulo $p$ because $(\forall\ b \in B)\,(\gcd{(b, p)} = 1)$, which will be called $b^{-1}$.  These inverses lie in the set $B$.  So there will be $\frac{p-3}{2}$ pairs of inverses because $p-1$ and $1$ are their own inverses from part a of this problem.  This implies the following,
\begin{align*}
	(p-2)! &\equiv\ 1\ \textrm{mod}\ p\\
	(p-1)(p-2)! &\equiv\ p-1\ \textrm{mod}\ p\\
	(p-1)! &\equiv\ -1\ \textrm{mod}\ p
\end{align*}

\paragraph{Problem 7c} ~\\
Suppose $n$ is a composite number.  So there are integers $a$ and $b$ such that $n=ab$.  This implies that $a < n$ and $b < n$, which means $a$ and $b$ will be in the product $(n-1)!$.  So $(n-1)! \equiv\ 0\ \textrm{mod} n$, and not $-1\ \textrm{mod}\ n$.

\paragraph{Problem 7d} ~\\
This primality test requires $n-2$ multiplications to compute $(n-1)!$.  This requires, $O(n(log_{2}n)^{2})$ bit operations.
 
\pagebreak
\paragraph{Problem 8a - Chinese Remainder Theorem} ~\\
\begin{table}[ht]
	\begin{tabular}{c c c}
	\hline\hline
	Number & modulo 5 & modulo 7\\ [0.5ex]
	\hline
	0 & 0 & 0 \\
	1 & 1 & 1 \\
	2 & 2 & 2 \\
	3 & 3 & 3 \\
	4 & 4 & 4 \\
	5 & 0 & 5 \\
	6 & 1 & 6 \\
	7 & 2 & 0 \\
	8 & 3 & 1 \\
	9 & 4 & 2 \\
	10 & 0 & 3 \\
	11 & 1 & 4 \\
	12 & 2 & 5 \\
	13 & 3 & 6 \\
	14 & 4 & 0 \\
	15 & 0 & 1 \\
	16 & 1 & 2 \\
	17 & 2 & 3 \\
	18 & 3 & 4 \\
	19 & 4 & 5 \\
	20 & 0 & 6 \\
	21 & 1 & 0 \\
	22 & 2 & 1 \\
	23 & 3 & 2 \\
	24 & 4 & 3 \\
	25 & 0 & 4 \\
	26 & 1 & 5 \\
	27 & 2 & 6 \\
	28 & 3 & 0 \\
	29 & 4 & 1 \\
	30 & 0 & 2 \\
	31 & 1 & 3 \\
	32 & 2 & 4 \\
	33 & 3 & 5 \\
	34 & 4 & 6 \\
	35 & 0 & 0 \\
	36 & 1 & 1
	\end{tabular}
\end{table}

\pagebreak
\paragraph{Problem 8b - Chinese Remainder Theorem} ~\\
Suppose $x$ and $y$ are two different prime numbers, and for every pair of integers $m$ and $n$, $0 \le m < x$ and $0 \le n < y$. \\\\
\noindent
Let A = $\{0, 1, \dots, xy - 1\}$. This is the range of $xy$.\\

\noindent
Since $0 \le m < x$ and $0 \le n < y$, it is known that $0 \le my < xy$ and $0 \le nx < xy$, which implies the following, $my \in A$ and $nx \in A$.  If $q$ is selected to be the following: 

\begin{equation*}
q = myy^{-1} + nxx^{-1} \text{, where $y^{-1}$ and $x^{-1}$ are inverses of $y$ mod $x$ and $x$ mod $y$, respectively.}
\end{equation*}
The inverses of $x$ and $y$ are defined because they are two different primes, making them relatively prime.  Then $q\ (\textrm{mod}\ xy) \in A$ by definition of the modulus operator, which allows the following, $0 \le q < 2xy \rightarrow 0 \le q\ (\textrm{mod}\ xy) < xy$. \\

\noindent
The next step is to show the following:
\begin{align*}
q &\equiv\ m (\textrm{mod}\ x) \\
q &\equiv\ n (\textrm{mod}\ y)
\end{align*}

\noindent
Using the selection of $q$ as the starting point, the integers $m$ and $n$ will be derived, mod $x$ and mod $y$, respectively.

\begin{align*}
q\ \textrm{mod}\ x &\equiv\ myy^{-1} + nxx^{-1} \equiv\ m mod x \text{, because $yy^{-1}$ is $1$ mod $x$ since they're inverses of each other, and $nxx^{-1}$ disappears.}\\
q\ \textrm{mod}\ y &\equiv\ myy^{-1} + nxx^{-1} \equiv\ n mod y \text {, because $xx^{-1}$ is $1$ mod $y$ since they're inverses of each other, and $myy^{-1}$ disappears.}
\end{align*}

\noindent
Now to prove the uniqueness of $q$.  Suppose there are two choices that satisfy the system above, $q_1, q_2 \in A$.  Then the following is true,

\begin{align*}
q_1 &\equiv\ m\ \textrm{mod}\ x\\
q_1 &\equiv\ n\ \textrm{mod}\ y\\
q_2 &\equiv\ m\ \textrm{mod}\ x\\
q_2 &\equiv\ n\ \textrm{mod}\ y\\
\\
q_1 - q_2 &\equiv\ 0\ \textrm{mod}\ x \rightarrow x\ |\ q_1-q_2\\
q_1 - q_2 &\equiv\ 0\ \textrm{mod}\ y \rightarrow y\ |\ q_1-q_2\\
\text{So $xy\ |\ q_1-q_2$}&\text{, which means $q_1 \equiv\ q_2$ mod $xy \rightarrow q_1=q_2$ because $q_1, q_2 \in A$.}
\end{align*}
\noindent
Therefore, $q$ is unique.

\paragraph{Problem 8c - Chinese Remainder Theorem} ~\\
Suppose $x$ and $y$ are different prime numbers such that,
\begin{align*}
q &\equiv\ m\ \textrm{mod}\ x\\
q &\equiv\ n\ \textrm{mod}\ y
\end{align*}

\noindent
Let $M_x=y$, $M_y=x$
\begin{align*}
&a_x \text{, be the inverse of $M_x$ mod $x$.}\\
&a_y \text{, be the inverse of $M_y$ mod $y$.}
\end{align*}
\noindent
So the follow equation for $q$ is derived,
\begin{equation*}
q = mM_xa_x + nM_ya_y\ \textrm{mod}\ xy
\end{equation*}
When $q$ is mod-ed with $x$, the second term $nM_ya_y$ disappears because $M_y=x$, and the part of the first term, $M_xa_x \equiv 1\ \textrm{mod}\ x$ because they are inverses of each other mod $x$. So $q \equiv\ m\ \textrm{mod}\ x$.\\

\noindent
When $q$ is mod-ed with $y$, the first term $mM_xa_x$ disappears because $M_x=y$, and the part of the second term, $M_ya_y \equiv 1\ \textrm{mod}\ y$ because they are inverses of each other mod $y$.  So $q \equiv\ n\ \textrm{mod}\ y$.\\

\paragraph{Problem 8d - Chinese Remainder Theorem} ~\\
In the case of three primes, $x$, $y$, and $z$, the property still holds.  When it is three primes the equation for $q$ changes to the following:
\begin{equation*}
q \equiv\ a_xM_xI_x + a_yM_yI_z + a_zM_zI_z\ \textrm{mod}\ M
\end{equation*}
where the parts of the equation are defined as followed:
Let $M=xyz$.
\begin{align*}
&a_x, a_y, a_z \text{, be the residues when mod-ed $x$, $y$, and $z$, respectively.}\\
&M_x=\frac{M}{x}=yz, M_y=\frac{M}{y}=xz, M_z=\frac{M}{z}=xy\\
&I_x, I_y, I_z \text{, be the inverses of $M_x$ mod $x$, $M_y$ mod $y$, and $M_z$ mod $z$, respectively.}
\end{align*}
\paragraph{Problem 9 - RSA Cryptography} ~\\
\begin{align*}
	\text{Let } &N_{b}, N_{c}, N_{d} \text{ be Bob, Charlie, and David's public key, respectively.}\\	
	&M=N_{b}N_{c}N_{d}\text{, }M_{b}=\frac{M}{N_{b}}\text{, }M_{c}=\frac{M}{N_{c}}\text{, }M_{d}=\frac{M}{N_{d}}\\
	&e\text{, be the encryption key for Bob, Charlie, and David.}\\
	&m_{a} \text{, be the message sent by Alice.}
\end{align*}

With the given information, the Chinese Remainder Theorem is applicable to find $m_{a}$:
\begin{align*}
e &= 3\\
M &=674 \cdot 36 \cdot 948 = 23002272\\
M_{b}&= 34128 \text{, } M_{c} = 638952 \text{, } M_{d}=24264\\
(m_{a})^{e} \equiv\ (m_{a})^{3}\ &\equiv\ 674\ \textrm{mod}\ N_{b}\ \equiv\ 674\ \textrm{mod}\ 3337\\
&\equiv\ 36\ \textrm{mod}\ N_{c}\ \equiv\ 36\ \textrm{mod}\ 187\\
&\equiv\ 948\ \textrm{mod}\ N_{d}\ \equiv\ 948\ \textrm{mod}\ 1219
\end{align*}
From the theorem, the equation we are looking for is:
\begin{equation}
(m_{a})^{3}\ \equiv\ (674)M_{b}y_{b} + (36)M_{c}y_{c} + (948)M_{d}y_{d}\ (mod M)
\end{equation}
The next step is to determine the inverses, $y_{b}$, $y_{c}$, and $y_{d}$.
\begin{itemize}
\item $M_{b}y_{b}\ \textrm{mod}\ N_{b}$, $y_{b}=2593$.
\item $M_{c}y_{c}\ \textrm{mod}\ N_{c}$, $y_{c}=90$.
\item $M_{d}y_{d}\ \textrm{mod}\ N_{d}$, $y_{d}=620$.
\end{itemize}
Going back to equation (1), insert the terms determined here:
\begin{align*}
(m_a)^{e} &\equiv\ (674)M_{b}y_{b} + (36)M_{c}y_{c} + (948)M_{d}y_{d}\ \textrm{(mod}\ M\textrm{)}\\
(m_a)^{3} &\equiv\ (674)(34128)(2593) + (36)(638952)(90) + (948)(24264)(620)\ \textrm{(mod}\ 23002272\textrm{)}\\
&\equiv\ 75976504416\ \textrm{(mod}\ 23002272 \textrm{)}\\
&\equiv\ 0\ \textrm{(mod}\ 23002272 \textrm{)}\\
m_a &\equiv\ 0\ \textrm{(mod}\ 23002272 \textrm{)}
\end{align*}

Therefore, the original message was $m_a = 0$.
\end{document}