\documentclass[fleqn]{article}
\usepackage{amsmath,amsfonts}
\usepackage{algorithm,algpseudocode}
\usepackage[margin=0.24in]{geometry}
\usepackage{enumitem}
\usepackage{tikz}
\usepackage{datetime}
\setdescription{leftmargin=\parindent,labelindent=\parindent}
\title{Homework 1 - Problem 5B}
\author{Mark Labrador}
\date{\today}

\begin{document}
\maketitle
\paragraph{Problem Statement:} ~\\
Consider the following sequence of numbers: $s_{n} = 1 + \prod\limits_{i=0}^{n-1} s_{i}$ , where $s_{0} = 2$. Prove that any two numbers in this sequence are relatively prime.

\paragraph{Proof by Induction:} ~\\

\paragraph{Basis:} ~\\
\begin{align*}
	s_{0} &= 2 \\
	s_{1} &= 1 + \prod\limits_{i=0}^{0} s_{i} = 1 + s_{0} = 1 + 2 = 3
\end{align*}

\paragraph{Induction Hypothesis: } ~\\
Suppose for all $n \ge 0$, $s_{n}$ are pairwise relatively prime.  In other words, $(\forall\, n \ge 1)\ \gcd\left(s_{n-1}, s_{n}\right)=1$.  Show that $\gcd\left(s_{n}, s_{n+1}\right)=1$.\\

\noindent
We know that the following property holds for gcd's, 

\begin{equation}\label{eq:trans_gcd}
\text{For some sequence of integers } a_{i}\text{, if } gcd(a_{i}, a_{i+1})=d \text{ and } gcd(a_{i+1}, a_{i+2})=d\text{, then } gcd(a_{i}, a_{i+2})=d.
\end{equation}

\noindent
For the sequence of numbers $s_{i}$, if we can show that $\gcd\left(s_{n}, s_{n+1}\right)=1$, then we'd have shown that all numbers in the sequence are relatively prime.

\begin{align*}
	s_{n+1} &= 1 + s_{n}\prod\limits_{i=0}^{n-1} s_{i} \\
	&= 1 + s_{n}\left(s_{n}-1\right) &\text{, because } s_{n}-1 = \prod\limits_{i=0}^{n-1} s_{i} \text{ from } s_{n} = 1 + \prod\limits_{i=0}^{n-1} s_{i} \text{, by definition.} \\
	1 &= s_{n+1} - \left(s_{n}-1\right)s_{n} \\
	&= as_{n+1} - bs_{n} &\text{, where $a=1$ and $b=s_{n} - 1$.}
\end{align*}

\noindent
So we have shown that $1$ is a linear combination of $s_n$ and $s_{n+1}$, and there is no other positive integer smaller than $1$ so we can conclude that $\gcd\left(s_n, s_{n+1}\right)=1$.  By the induction hypothesis and property \ref{eq:trans_gcd}, all numbers in the sequence $s_{i}$ for $i=0 \dots n$ are pairwise relatively prime.

\paragraph{QED}

\end{document}